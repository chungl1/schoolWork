\documentclass[12pt]{article}

\usepackage{graphicx}
\usepackage{paralist}
\usepackage{listings}
\usepackage{booktabs}

\oddsidemargin 0mm
\evensidemargin 0mm
\textwidth 160mm
\textheight 200mm

\pagestyle {plain}
\pagenumbering{arabic}

\newcounter{stepnum}

\title{Assignment 2 Report}
\author{Lawrence Chung, chungl1}
\date{\today}

\begin {document}

\maketitle

This report covers information about testing of the original and partner's code. The report 
also reveals any ambiguous points of the assignment that would need fixing if ever used in
the industry.

\section{Testing of the Original Program}

Testing involved two categories: Tests that would throw exceptions and tests that would not.
The tests that did not throw exceptions were judged as correct if the expected output occurred
and inaccurate if expected output did not occur. For exceptions, Pytest classified those test cases
as errors. One problem uncovered through testing was that the specification was not clear about formatting
floating point values. This problem will be further discussed in the report.\\

All testing followed the same criteria. Failed tests are tests that gave unexpected output or threw 
exceptions. Successful test are tests that output the expected values.\\

\section{Results of Testing Partner's Code}

Most of the results of my partner's code was expected. That is, exceptions were thrown when expected, but 
some methods behaved differently in comparison to the original code. There were a total of 13 failed
tests, and 11 passed tests. \\

\section{Discussion of Test Results}

The results of the original code yielded 10 failed tests and 14 passed tests. Most of the testing resulted
in expected output with the exception of a few. The unexpected output will be further discussed later in
the report.\\

\subsection{Problems with Original Code}

The unexpected output stemmed from, as mentioned earlier, CurveT.eval, CurveT.dfdx and Curve.df2dx2.
The reason for this unexpected output was the lack of clarity in the specification about formatting
of floating point values. For example, if an expected value was 4, the output would be 3.99999999.
The value isn't necessarily incorrect, especially in the context of this assignment, but because 
there was no specification about formatting, this result is unexpected and was deemed as a failed test.\\

\subsection{Problems with Partner's Code}

Some of my partner's unexpected output stemmed from the same issue of formatting. However, there were 
some other cases of unexpected output. There were some attribute errors that were unexpected that prevented
Data.eval from running. Another error was from SeqServices.index, where the output was not expected and led 
to an assertion error. The last error that was different from the original code was from a test for 
non-Ascending sequence for SeqServices.index. However, my partner code was still correct because there was
no mention about exception handling for that method.\\

\subsection{Problems with Assignment Specification}

The main issue with the Assignment specification is the formatting of the floating point values. In comparison
to assignment 1, the specification is more formal, but the Load.py and Plot.py would require a more formal 
specification. It was explained in words, so the potential for ambiguity still exists. \\

\section{Answers}

\begin{enumerate}

\item What is the mathematical specification of the \texttt{SeqServices} access
  program isInBounds(X, x) if the assumption that X is ascending is removed?\\

  $out := (\lnot isAscending \Rightarrow IndepVarNotAscending | X_0 \leq x \leq X_{|X|-1})$

\item How would you modify \texttt{CurveADT.py} to support cubic interpolation?\\

In the constructor, change the exception to only occur if order is greater than 3.\\

\item What is your critique of the CurveADT module's interface.  In particular,
  comment on whether the exported access programs provide an interface that is
  consistent, essential, general, minimal and opaque.\\

  The module interface is essential.\\
  \\
  It is consistent to a high degree, with a few exceptions. The method signatures are the
  same, but performance is not on par across different code.\\
  \\
  The interface is general, in that it cannot be predicted how it will be used. In this 
  assignment, it is used to plot graphs, but it has other characteristics that allow it
  to be used for another purpose. \\
  \\
  The interface is minimal. Each method in the ADT performs one task for the user.\\
  \\
  The interface is not as opaque as it should be. It does not mention anything specific
  about which variables should not be accessible to the user.\\

\item What is your critique of the Data abstract object's interface.  In
  particular, comment on whether the exported access programs provide an
  interface that is consistent, essential, general, minimal and opaque.\\

  In terns of consistency, essentiality, generality, minimality and opaqueness, Data 
  and CurveADT are similar. Main difference is that Data.py is more general than CurveADT.
  Data provides many methods purely on data manipulation, but data manipulation can be 
  used in many contexts. 

\end{enumerate}

\newpage

\lstset{language=Python, basicstyle=\tiny, breaklines=true, showspaces=false,
  showstringspaces=false, breakatwhitespace=true}

\def\thesection{\Alph{section}}

\section{Code for CurveADT.py}

\noindent \lstinputlisting{../src/CurveADT.py}

\newpage

\section{Code for Data.py}

\noindent \lstinputlisting{../src/Data.py}

\newpage

\section{Code for SeqServices.py}

\noindent \lstinputlisting{../src/SeqServices.py}

\newpage

\section{Code for Plot.py}

\noindent \lstinputlisting{../src/Plot.py}

\newpage

\section{Code for Load.py}

\noindent \lstinputlisting{../src/Load.py}

\newpage

\section{Code for Partner's CurveADT.py}

% Uncomment the line below when partner files have been pushed to your repo
\noindent \lstinputlisting{../partner/CurveADT.py}

\newpage

\section{Code for Partner's Data.py}

% Uncomment the line below when partner files have been pushed to your repo
\noindent \lstinputlisting{../partner/Data.py}

\newpage

\section{Code for Partner's SeqServices.py}

% Uncomment the line below when partner files have been pushed to your repo
\noindent \lstinputlisting{../partner/SeqServices.py}

\newpage

\section{Makefile}

\lstset{language=make}
\noindent \lstinputlisting{../Makefile}

\end {document}
